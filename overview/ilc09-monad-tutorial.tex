%-----------------------------------------------------------------------------
%
%               Template for LaTeX Class/Style File
%
% Name:         sigplanconf-template.tex
% Purpose:      A template for sigplanconf.cls, which is a LaTeX 2e class
%               file for SIGPLAN conference proceedings.
%
% Author:       Paul C. Anagnostopoulos
%               Windfall Software
%               978 371-2316
%               paul@windfall.com
%
% Created:      15 February 2005
%
%-----------------------------------------------------------------------------


\documentclass[preprint,natbib,10pt]{sigplanconf}

\usepackage{amsmath}

\begin{document}

\conferenceinfo{ILC '09}{March 22--25, 2009, Cambridge, Massachusetts, USA.}
\copyrightyear{2009}
\copyrightdata{[to be supplied]}

\titlebanner{DRAFT}        % These are ignored unless
\preprintfooter{A tutorial on using monads in Clojure and scheme}   % 'preprint' option specified.

\title{Monads for the Working List Programmer}
% \subtitle{Subtitle Text, if any}

\authorinfo{Ravi Nanavati\and Jeff Newbern}
           {Bluespec, Inc.}
           {\{ravi,jnewbern\}@bluespec.com}

\maketitle

\begin{abstract}

Monads are a computational pattern for encapsulating and controlling
effects.  First employed in pure languages, monad design patterns have
recently been implemented in impure functional languages.  This
tutorial introduces monads as implemented in the pure language Haskell
before describing monad facilities in scheme and Clojure.  An extended
example of monadic programming in Clojure is developed by building a
modular language interpreter.

\end{abstract}

\category{D.1.1}{PROGRAMMING TECHNIQUES}{Applicative (Functional) Programming}

%\terms
%term1, term2

%\keywords
%keyword1, keyword2

\section{Introduction}

The monad design pattern adds structure to a value so that effects
associated with computing the value can be modeled.  A wide variety of
effects can be modeled this way, including things that are not
normally thought of as effects, such as parser failure or accessing an
environment of bindings.

For the programmer, monads are useful tools for structuring functional
programs because they provide modularity, flexibility and isolation.
Modularity is achieved because monads allow effectful computations to
be composed from simpler effectful computations.  Programs built using
monads can be more flexible than non-monadic programs because the
specification of the strategy for combining effectful computations is
factored out into a single code fragment that can be easily modified
or replaced.  Monads make it easier to isolate effectful computations
and to track and delimit the scope of effects within a program.

\subsection{Definition}

There are three components to a computational monad:
\begin{enumerate}
\item
Some additional structure around a base value which will model the
effect of the monad.  We represent the type of the base value by
$t$ and the type of the monadic value which extends the base type
with a model of the desired effect by writing the name of the effect
in a subscript, $t_{effect}$.

\item
A function, {\tt result}, of type $t \rightarrow t_{effect}$) which
takes a value of the base type $t$ and lifts it to a value of the
monadic type by adding structure corresponding to a pure computation
with no effect.

\item
A function, {\tt bind}, of type $t_{effect} \rightarrow (t \rightarrow
t_{effect}) \rightarrow t_{effect}$ which combines a monadic value
with a function to create a new monadic value and combines the effects
of the two.  The values of the base type in the result of {\tt bind}
are computed from the values of the base type in the input value by
applying the input function.  The effect structure of the result of
{\tt bind} is the composition of the effect structure of the input
value with the effect structure of the result of the function
application.
\end{enumerate}

Since the {\tt bind} function only composes existing effect structures
and the {\tt result} function lifts a value purely -- that is without
adding an effect -- any useful non-identity monad will be augmented
with additional functions which create monadic values with effects.
Unlike {\tt result} and {\tt bind}, the names and types of these
functions are not standardized and will vary between different monads.

\subsection{Monad Laws}

The {\tt result} and {\tt bind} functions together describe a system
for composing computations with effects.  To ensure that a monad
implements a consistent effect system, the monad should obey three laws:

\begin{enumerate}
\item {\tt bind (result x) f $\equiv$ f x}
\item {\tt bind m result $\equiv$ m}
\item {\tt bind (bind m f) g $\equiv$ bind m ($\lambda$x $\rightarrow$ bind (f x) g)}
\end{enumerate}

The first law requires that {\tt result} is a left-identity with
respect to {\tt bind}. The second law requires that {\tt result} is a
right-identity with respect to {\tt bind}. The third law is a kind of
associativity law for {\tt bind}.

\subsection{Monad Transformers}

Monad transformers allow monads for different effects to be composed
into a combined monad.  Multiple monad transformers can be applied to
a base monad to create a ``transformer stack'' that combines the
effects of each individual monad in the stack.  In general monad
composition is not commutative, so the order in which transformers are
applied in the stack affects the semantics of the combined monad.

\section{Monads in Haskell}

In the pure, non-strict, statically-typed functional language Haskell
monads are pervasive.  Impure features such as I/O and mutable state
are built on a monadic framework and monads are used to model failure,
nondeterminism, partiality, parsing, and numerous other effects.

In Haskell a monad is represented by a constructor for the type of
monadic values given the type of the base value, a {\tt retur}
function for lifting values of the base type to the monadic type, and
an infix binding combinator, {\tt >>=}, for composing a value of the
monadic type with a function to create a new value of the monadic
type, combining the effects of the input value and the function.

\begin{figure}
\begin{center}
\begin{verbatim}
class Monad m where
    (>>=)  :: m a -> (a -> m b) -> m b
    return :: a -> m a
\end{verbatim}
\end{center}
\caption{Haskell's Monad class}
\label{fig-Haskell-Monad-class}
\end{figure}

Haskell's {\tt return} function represents {\tt result} and {\tt >>=}
represents {\tt bind}.

Haskell provides a library of standard monads and monad transformers,
including transformers for working with state, continuations, failure,
environments, logging and errors.  In addition, users can easily
define their own monads and monad transformers with work seamlessly
with Haskell's built in notation and features for working with monads.

\section{Monads in Scheme}

Scheme does not have native support for monads, but a number of
different packages have been developed for working with monads.  No
package has reached the level of a de-facto standard.

\section{Monads in Clojure}

Clojure has a growing library of contributed code for working with
monads patterned after Haskell's monad libraries in its
clojure.contrib.monads library.

Monads are represented by structures with standard field names
{\tt m-result} and {\tt m-bind} and the library provides a number of
macros for working with them.

The {\tt with-monad} macro takes a monad structure and a number of
expressions as arguments and binds the {\tt m-bind} and {\tt m-result}
fields of the monad structure to local names so that any occurence of
{\tt m-bind} or {\tt m-result} in the expressions will refer to the
fields of the selected monad.  The expressions are said to be ``run
in'' the selected monad.  Often monadic code can only be run in a
single monad, but it sometimes possible to write code which can run in
different monads to achieve different effects (eg. running in either
an interpretation monad or a compilation monad).  The definition:
\begin{verbatim}
(defn lookup [key]
  (with-monad maybe
    (let [entry (find db key)]
       (if entry
           (m-result (second entry))
           m-zero))))
\end{verbatim}
shows a simple example of {\tt with-monad} used to run a computation
in the {\tt maybe} monad.

The {\tt domonad} macro is used for monad comprehensions with a list
of monadic bindings and a final expression.  Each of the bindings is
evaluated in turn, making the value accessible to all later bindings
and the final expression.  The final result of the {\tt domonad}
comprehension is the value of the final expression returned using
the monad's {\tt m-result} function.

For example, the comprehension:
\begin{verbatim}
(domonad maybe
  ; bindings
  [record (lookup employee)]
  ; final expression
  (:last-name record))
\end{verbatim}
is equivalent to the unsugared form:
\begin{verbatim}
(bind (lookup employee)
      (fn [record]
        (m-result (:last-name record))))
\end{verbatim}

The {\tt defmonad} macro allows definition of user-defined named
macros.  It also used internally to define the monads supplied
with the library, such as the {\tt maybe} monad.

\begin{figure}
\begin{center}
\begin{verbatim}
(defmonad maybe
   "Monad describing computations with possible
    failures. Failure is represented by nil,
    any other value is considered valid.
    As soon as a step returns nil, the whole
    computation will yield nil as well."
   [m-zero   nil
    m-result (fn m-result-maybe [v] v)
    m-bind   (fn m-bind-maybe [mv f]
               (if (nil? mv) nil (f mv)))
    m-plus   (fn m-plus-maybe [& mvs]
               (first (drop-while nil? mvs)))
    ])

\end{verbatim}
\end{center}
\caption{Clojure's maybe monad definition}
\label{fig-Clojure-maybe-defn}
\end{figure}



\section{A Modular Interpreter}

%\appendix
%\section{Appendix Title}

%This is the text of the appendix, if you need one.

%\acks

%Acknowledgments, if needed.

\bibliographystyle{plainnat}

\begin{thebibliography}{}

\bibitem{smith02}
Smith, P. Q. reference text

\end{thebibliography}

\end{document}
