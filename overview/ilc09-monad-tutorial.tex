%-----------------------------------------------------------------------------
%
%               Template for LaTeX Class/Style File
%
% Name:         sigplanconf-template.tex
% Purpose:      A template for sigplanconf.cls, which is a LaTeX 2e class
%               file for SIGPLAN conference proceedings.
%
% Author:       Paul C. Anagnostopoulos
%               Windfall Software
%               978 371-2316
%               paul@windfall.com
%
% Created:      15 February 2005
%
%-----------------------------------------------------------------------------


\documentclass[preprint,natbib,10pt]{sigplanconf}

\usepackage{amsmath}

\begin{document}

\conferenceinfo{ILC '09}{March 22--25, 2009, Cambridge, Massachusetts, USA.}
\copyrightyear{2009}
\copyrightdata{[to be supplied]}

\titlebanner{DRAFT}        % These are ignored unless
\preprintfooter{A tutorial on using monads in Clojure and scheme}   % 'preprint' option specified.

\title{Monads for the Working List Programmer}
% \subtitle{Subtitle Text, if any}

\authorinfo{Ravi Nanavati\and Jeff Newbern}
           {Bluespec, Inc.}
           {\{ravi,jnewbern\}@bluespec.com}

\maketitle

\begin{abstract}

Monads are a computational pattern for encapsulating and controlling
effects.  First employed in pure languages, monad design patterns have
recently been implemented in impure functional languages.  This
tutorial introduces monads as implemented in the pure language Haskell
before describing monad facilities in scheme and Clojure.  An extended
example of monadic programming in Clojure is developed by building a
modular language interpreter.

\end{abstract}

\category{D.1.1}{PROGRAMMING TECHNIQUES}{Applicative (Functional) Programming}

%\terms
%term1, term2

%\keywords
%keyword1, keyword2

\section{Introduction}

The monad design pattern adds structure to a value so that effects
associated with computing the value can be modeled.  A wide variety of
effects can be modeled this way, including things that are not
normally thought of as effects, such as parser failure or accessing an
environment of bindings.

For the programmer, monads are useful tools for structuring functional
programs because they provide modularity, flexibility and isolation.
Modularity is achieved because monads allow effectful computations to
be composed from simpler effectful computations.  Programs built using
monads can be more flexible than non-monadic programs because the
specification of the strategy for combining effectful computations is
factored out into a single code fragment that can be easily modified
or replaced.  Monads make it easier to isolate effectful computations
and to track and delimit the scope of effects within a program.

There are 3 components to a computational monad:

Some additional structure around a base value which will model the
effect of the monad.  We represent the type of the base value by
$t$ and the type of the monadic value which extends the base type
with a model of the desired effect by writing the name of the effect
in a subscript, $t_{effect}$.

A function, {\tt result}, of type $t \rightarrow t_{effect}$) which
takes a value of the base type $t$ and lifts it to a value of the
monadic type by adding structure corresponding to a pure computation
with no effect.

A function, {\tt bind}, of type $t_{effect} \rightarrow (t \rightarrow
t_{effect}) \rightarrow t_{effect}$ which combines a monadic value
with a function to create a new monadic value and combines the effects
of the two.  The values of the base type in the result of {\tt bind}
are computed from the values of the base type in the input value by
applying the input function.  The effect structure of the result of
{\tt bind} is the composition of the effect structure of the input
value with the effect structure of the result of the function
application.

Since the {\tt bind} function composes existing effect structures into
its result effect structure and the {\tt result} function lifts a
value purely -- that is without adding an effect -- any useful
non-identity monad will be augmented with additional functions which
create monadic values with effects.  Unlike {\tt return} and
{\tt bind}, the names and types of these functions are not standardized
and will vary between different monads.

<<< monad laws>>>

\section{Monads in Haskell}

In the pure, non-strict, statically-typed functional language Haskell
monads are pervasive.  Impure features such as I/O and mutable state
are built on a monadic framework and monads are used to model failure,
nondeterminism, partiality, and numerous other effects.

In Haskell a monad is represented by a constructor for the type of
monadic values given the type of the base value, a {\tt result}
function for lifting values of the base type to the monadic type, and
a binding combinator, {\tt >>=}, for composing a value of the monadic
type with a function to create a new value of the monadic type,
combining the effects of the input value and the function.

\section{Monads in Scheme}

\section{Monads in Clojure}

\section{A Modular Interpreter}

%\appendix
%\section{Appendix Title}

%This is the text of the appendix, if you need one.

%\acks

%Acknowledgments, if needed.

\bibliographystyle{plainnat}

\begin{thebibliography}{}

\bibitem{smith02}
Smith, P. Q. reference text

\end{thebibliography}

\end{document}
