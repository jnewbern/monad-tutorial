%-----------------------------------------------------------------------------
%
%               Template for LaTeX Class/Style File
%
% Name:         sigplanconf-template.tex
% Purpose:      A template for sigplanconf.cls, which is a LaTeX 2e class
%               file for SIGPLAN conference proceedings.
%
% Author:       Paul C. Anagnostopoulos
%               Windfall Software
%               978 371-2316
%               paul@windfall.com
%
% Created:      15 February 2005
%
%-----------------------------------------------------------------------------


\documentclass[preprint,natbib,10pt]{sigplanconf}

\usepackage{amsmath}

\begin{document}

\conferenceinfo{ILC '09}{March 22--25, 2009, Cambridge, Massachusetts, USA.}
\copyrightyear{2009}
\copyrightdata{[to be supplied]}

\titlebanner{DRAFT}        % These are ignored unless
\preprintfooter{A tutorial on using monads in Clojure and scheme}   % 'preprint' option specified.

\title{Monads for the Working List Programmer}
% \subtitle{Subtitle Text, if any}

\authorinfo{Ravi Nanavati\and Jeff Newbern}
           {Bluespec, Inc.}
           {\{ravi,jnewbern\}@bluespec.com}

\maketitle

\begin{abstract}

Monads are a computational pattern for encapsulating and controlling
effects.  First employed in pure languages, monad design patterns have
recently been implemented in impure functional languages.  This
tutorial introduces monads as implemented in the pure language Haskell
before describing monad facilities in scheme and Clojure.  An extended
example of monadic programming in Clojure is developed by building a
modular language interpreter.

\end{abstract}

\category{D.1.1}{PROGRAMMING TECHNIQUES}{Applicative (Functional) Programming}

\terms
term1, term2

\keywords
keyword1, keyword2

\section{Introduction}

\section{Monads in Haskell}

\section{Monads in Scheme}

\section{Monads in Clojure}

\section{A Modular Interpreter}

\appendix
\section{Appendix Title}

This is the text of the appendix, if you need one.

\acks

Acknowledgments, if needed.

\bibliographystyle{plainnat}

\begin{thebibliography}{}

\bibitem{smith02}
Smith, P. Q. reference text

\end{thebibliography}

\end{document}
